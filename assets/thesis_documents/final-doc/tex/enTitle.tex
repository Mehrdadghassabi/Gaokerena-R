% !TeX root=../main.tex
% در این فایل، عنوان پایان‌نامه، مشخصات خود و چکیده پایان‌نامه را به انگلیسی، وارد کنید.

%%%%%%%%%%%%%%%%%%%%%%%%%%%%%%%%%%%%
\latinuniversity{University of Isfahan}
\latinfaculty{Faculty of Computer Engineering}
\latindepartment{Artificial Intelligence and robotic}
\latinsubject{Computer Engineering}
\latinfield{Artificial Intelligence and robotic}
\latintitle{Developing a medical language model based on reasoning in Persian language}
\firstlatinsupervisor{First Supervisor}
\secondlatinsupervisor{Second Supervisor}
\firstlatinadvisor{First Advisor}
%\secondlatinadvisor{Second Advisor}
\latinname{Mehrdad}
\latinsurname{Ghassabi}
\latinthesisdate{January 2026}
\latinkeywords{Artificial Intelligence in medicine,Persian language models,Medical language models,Natural language processing,Artificial Intelligence reasoning}
\en-abstract{
The use of artificial intelligence in answering medical questions is recognized as one of the emerging and critical fields in technology and healthcare, which has garnered widespread attention in recent years. This advanced technology, with its unique capabilities, can significantly enhance the quality of medical services provided to patients. Additionally, by accelerating the process of delivering medical information and providing quick and accurate responses to the questions of doctors and patients, it plays a vital role in reducing the workload of physicians. As such, artificial intelligence not only increases efficiency in healthcare systems but also improves the overall patient experience and paves the way for better and more effective treatments.
On the other hand, since medicine is based on reasoning and logical analysis, developing a medical model designed on a chain of logical thoughts and reasoning can significantly enhance the accuracy and efficiency of such a model. This approach facilitates the execution of complex diagnostic and therapeutic processes in a more structured and purposeful manner. In this regard, every stage of diagnosis and treatment must be based on scientific evidence and reliable data. For instance, in the process of diagnosing diseases, doctors typically rely on medical history, clinical symptoms, and test results. By designing a logical model, these data can be interconnected in a logical chain, helping identify patterns and relationships between symptoms and diseases.
}
