% !TeX root=../main.tex
\chapter{بررسی کار های پیشین}
\section{مقدمه}
همان‌طور که پیش‌تر اشاره شد، علیرغم پیشرفت‌های چشمگیر در توسعه مدل‌های زبانی پزشکی به زبان انگلیسی، مانند توسعه و معرفی مدل‌های
\lr{MedPalm}
\cite{b13}
\cite{b14}
یا مدل
\lr{Med-Gemini}
\cite{b15}
،متأسفانه در حوزه زبان فارسی هنوز کار چندانی در این زمینه انجام نشده است. این مسئله بدین معناست که ما در حوزه زبان فارسی تقریباً با یک کاغذ سفید روبه‌رو هستیم. در این پایان‌نامه تلاش شده است تا قدمی رو به جلو در جهت توسعه مدل‌های زبانی پزشکی برای زبان فارسی برداشته شود.

در ادامه، به بررسی کارهای پیشین انجام‌شده، چه در حوزه زبان فارسی و چه در حوزه زبان انگلیسی، خواهیم پرداخت.
\section{کار های پیشین در حوزه زبان انگلیسی}
\subsection{مدل های
	\lr{Med-Palm}
}
مدل های 
\lr{med-palm}
یکی از مدل‌های زبانی پزشکی بزرگ 
\footnote{\lr{large medical language models}}
است که توسط تیم تحقیقاتی گوگل برای کاربرد های پزشکی توسعه داده شده است. این مدل با استفاده از داده‌های تخصصی پزشکی و بالینی آموزش دیده است. هدف اصلی این خانواده از مدل های زبانی پزشکی پاسخ‌گویی به پرسش‌های پزشکی با دقت بالا، کمک به پزشکان در تصمیم‌گیری‌های بالینی، و تسهیل دسترسی به اطلاعات پزشکی برای کاربران است. نسخه‌های مختلف این مدل، مانند
\lr{MedPaLM}
و
\lr{MedPaLM2}
، توانایی‌های قابل توجهی در درک و تحلیل زبان تخصصی پزشکی نشان داده‌اند و به عنوان یک ابزار نوین در حوزه هوش مصنوعی پزشکی شناخته می‌شوند. این مدل‌ها با استفاده از آزمون‌های استاندارد پزشکی (مانند
\lr{USMLE}) 
ارزیابی شده و توانسته‌اند عملکردی نزدیک به سطح متخصصین پزشکی ارائه دهند. مدل
\lr{MedPaLM2}
به عنوان یک گام مهم در جهت توسعه مدل‌های زبان تخصصی در حوزه سلامت و پزشکی شناخته می‌شود.
\subsection{مدل 
	\lr{ChatDoctor}
}
مدل
\lr{ChatDoctor}
\cite{b16}
یکی از برجسته‌ترین تلاش‌ها در حوزه توسعه مدل‌های زبانی پزشکی است که شباهت قابل توجهی به فاز نخست پایان‌نامه حاضر دارد. تیم توسعه‌دهنده این مدل، داده‌های آموزشی خود را از دو پلتفرم آنلاین پرسش و پاسخ پزشکی به نام‌های
\lr{HealthcareMagic}
و
\lr{iCliniq}
جمع‌آوری کرده‌اند. این تیم ابتدا بیش از دویست هزار جفت پرسش و پاسخ پزشکی از این منابع گردآوری کرده و سپس با اعمال فیلترهایی بر اساس طول و کیفیت پاسخ‌ها، مجموعه‌ای با کیفیت بالا شامل صد هزار جفت پرسش و پاسخ نهایی ایجاد کرده‌اند. داده‌های مذکور به‌عنوان پایه‌ای برای آموزش و تنظیم دقیق
\footnote{\lr{fine-tuning}}
مدل
\lr{LLaMa}
\cite{b17}
مورد استفاده قرار گرفته‌اند تا مدلی توانمند در تولید اطلاعات پزشکی دقیق و مرتبط ایجاد شود.

علاوه بر این، این مدل از رویکرد تولید مبتنی بر بازیابی اطلاعات
\footnote{\lr{Retrieval-Augmented Generation (RAG)}}
بهره برده است. این رویکرد به مدل امکان می‌دهد تا به اطلاعات جدید و خارجی دسترسی پیدا کرده و آن‌ها را به‌طور مؤثر در پاسخ‌های خود ادغام کند. چنین رویکردی موجب ارتقای عملکرد کلی سیستم شده و توانایی مدل در تولید پاسخ‌هایی دقیق‌تر و مرتبط‌تر را به‌طور چشمگیری بهبود بخشیده است.

\subsection{مدل های 
	\lr{Meerkat}
}
مدل‌های
\lr{Meerkat}
\cite{b18}
یکی دیگر از تلاش‌های برجسته در حوزه توسعه مدل‌های زبانی پزشکی است. این پروژه با استخراج زنجیره‌های تفکر
\footnote{\lr{chain of thought}}
از کتاب‌های درسی پزشکی و تنظیم دقیق یک مدل زبانی پایه با استفاده از این داده‌ها، همراه با مجموعه داده‌های مکمل دیگر، به وجود است.همانند فاز دوم پایان نامه حاضر هدف اصلی
\lr{Meerkat}
تمرکز بر فرآیندهای استدلالی است که در تصمیم‌گیری‌های پزشکی نقش دارند. این مدل تلاش کرده است تا نه تنها اطلاعات پزشکی دقیق ارائه دهد، بلکه فرآیندهای شناختی و تصمیم‌گیری متخصصان حوزه سلامت را شبیه‌سازی کند. به همین دلیل، 
\lr{Meerkat}
به عنوان مدلی برای تعاملات پیچیده‌تر و آگاهانه‌تر در حوزه پزشکی معرفی شده است.
\subsection{مدل 
	\lr{MedMobile}
}
\lr{MedMobile}
\cite{b19}
تلاشی دیگر در حوزه مدل‌های زبانی کوچک پزشکی است. برای توسعه این مدل زبانی کوچک، مدل
\lr{Phi-3-mini}
\cite{b20}
به عنوان مدل پایه
\footnote{\lr{baseline model}}
استفاده از ترکیبی از داده‌های مصنوعی و تولیدشده توسط انسان تنظیم دقیق
\footnote{\lr{fine tune}}
 شده است تا عملکردی بهینه و مناسب برای اجرا روی دستگاه‌های همراه مانند موبایل ارائه دهد. با تمرکز بر نیازهای خاص کاربران دستگاه های همراه، 
\lr{MedMobile}
تلاش کرده است مدلی کارآمد و مؤثر فراهم کند که دسترسی به اطلاعات پزشکی باکیفیت را در هر زمان و مکان به صورت محلی 
\footnote{\lr{local}}
ممکن می‌سازد.
\section{کار های پیشین در حوزه زبان فارسی}
همان‌طور که پیش‌تر اشاره شد، تحقیقات محدودی بر روی مدل‌های زبانی پزشکی فارسی تمرکز داشته‌اند که این امر نشان‌دهنده شکاف قابل توجهی در منابع موجود برای جامعه پزشکی فارسی‌زبان است. علاوه بر این، پژوهش‌های بسیار اندک موجود در این زمینه ، به طور کامل در مورد مجموعه داده‌ها، مدل‌ها و کدهای خود متن بسته
\footnote{\lr{closed-source}}
هستند.

از سوی دیگر، تمامی این تلاش‌ها عمدتا بر روی راهکارهای استخراجی 
\footnote{\lr{extractive}}
متمرکز بوده‌اند که هدفشان بازیابی اطلاعات مرتبط از منابع از پیش تعریف شده است، به جای استفاده از رویکردهای تولیدی
\footnote{\lr{generative}}
که قادر به تولید پاسخ‌های آگاه از زمینه باشند.
\subsection{مدل 
	\lr{Sina-bert}
}
شاید اولین و برجسته ترین مدل زبانی پزشکی فارسی، 
\lr{Sina-BERT}
\cite{b21}
باشد که شامل آموزش یک مدل
\lr{BERT}
\cite{b22}
با استفاده از یک پیکره خزش شده
\footnote{\lr{crawled}}
همراه با مجموعه‌داده پرسش و پاسخ پزشکی فارسی است که به طور خاص برای کاربردهای مختلف از جمله پاسخ به سوالات پزشکی، تحلیل احساسات پزشکی و بازیابی سوالات پزشکی توسعه یافته‌اند.

\lr{Sina-BERT}
در میان تلاش‌های متمرکز بر زبان فارسی، بیشترین شباهت را به فاز نخست پایان نامه حاضر دارد؛ با این تفاوت که از مدل برت
\footnote{\lr{BERT}}
یک مدل زبانی مبتنی بر رمزگذار
\footnote{\lr{encoder-based}}
به عنوان مدل پایه استفاده می‌کند. این انتخاب تولید پاسخ توسط این مدل را عملا ناممکن می سازد، چرا که برت عمدتا برای درک و استخراج اطلاعات طراحی شده است نه برای تولید پاسخ.


\subsection{
	سیستم پرسش و پاسخ پزشکی دکتر ویسی و همکاران
}
یکی از آثار برجسته در حوزه پردازش زبان طبیعی، سیستم پرسش و پاسخ پزشکی فارسی است که توسط دکتر ویسی و همکارانش 
\cite{b23}
طراحی و توسعه داده شده است. این سیستم به‌طور کلی شامل سه ماژول اصلی است: پردازش پرسش، بازیابی سند و استخراج پاسخ. ماژول پردازش پرسش وظیفه تحلیل و اصلاح پرسش‌های کاربران را برعهده دارد تا پرسش‌ها به شکل بهینه برای مراحل بعدی آماده شوند. سپس، ماژول بازیابی سند با استفاده از الگوریتم‌های پیشرفته، اسناد پزشکی مرتبط را از میان داده‌های از پیش تعیین‌شده پیدا می‌کند. در نهایت، ماژول استخراج پاسخ با شناسایی دقیق اطلاعات موجود در اسناد بازیابی‌شده، مناسب‌ترین پاسخ‌ها را استخراج کرده و به کاربران ارائه می‌دهد. این سیستم نه تنها به‌طور مؤثر به پرسش‌های پزشکی پاسخ می‌دهد، بلکه ساختار ماژولار آن امکان بهبود و توسعه در آینده را نیز فراهم می‌سازد.
\subsection{
	پایان نامه کارشناسی ارشد خانم لیلا دارابی
}
مشابه به این دو اثر، پیشین لیلا دارابی در پایان نامه ارشد خود
\cite{b24}
از مدل‌هایی مانند 
\lr{Pars-BERT}
\cite{b25}
برای بازیابی پاسخ‌های مرتبط استفاده کرده است. رویکرد او شامل یافتن سوالات مشابه برای مدیریت پرسش‌های تکراری و به کارگیری استراتژی‌های ارزیابی دقیق و سهل‌گیرانه برای پاسخ‌های دقیق یا تقریبی می‌شود. علاوه بر این، روش‌های طبقه‌بندی و شناسایی موجودیت‌های نامدار
\footnote{Named Entity Recognition (NER)}
برای بهبود ارتباط پاسخ‌ها از طریق دسته‌بندی سوالات و شناسایی موجودیت‌های پزشکی مانند نام داروها و بیماری‌ها به کار گرفته می‌شوند.
