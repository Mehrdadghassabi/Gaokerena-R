% !TeX root=../main.tex
% در این فایل، عنوان پایان‌نامه، مشخصات خود، متن تقدیمی‌، ستایش، سپاس‌گزاری و چکیده پایان‌نامه را به فارسی، وارد کنید.
% توجه داشته باشید که جدول حاوی مشخصات پروژه/پایان‌نامه/رساله و همچنین، مشخصات داخل آن، به طور خودکار، درج می‌شود.
%%%%%%%%%%%%%%%%%%%%%%%%%%%%%%%%%%%%
% دانشگاه خود را وارد کنید
\university{دانشگاه اصفهان}
% پردیس دانشگاهی خود را اگر نیاز است وارد کنید (مثال: فنی، علوم پایه، علوم انسانی و ...)
\college{پردیس دانشکده‌های فنی}
% دانشکده، آموزشکده و یا پژوهشکده  خود را وارد کنید
\faculty{دانشکده مهندسی کامپیوتر}
% گروه آموزشی خود را وارد کنید (در صورت نیاز)
\department{هوش مصنوعی و رباتیکز}
% رشته تحصیلی خود را وارد کنید
\subject{هوش مصنوعی و رباتیکز}
% گرایش خود را وارد کنید
\field{هوش مصنوعی}
% عنوان پایان‌نامه را وارد کنید
\title{
‫ﺗﻮﺳﻌﻪ ‬‫ﯾﮏ ‬‫ﻣﺪل‬ ‫زﺑﺎﻧﯽ‬ ‫ﭘ‬ﺰﺷﮑﯽ‬ ‫ﻣﺒﺘﻨﯽ‬ ‫ﺑﺮ‬ ‫اﺳﺘﺪﻻل در زبان فارسی
}
% نام استاد(ان) راهنما را وارد کنید
\firstsupervisor{دکتر حمیدرضا برادران کاشانی}
\firstsupervisorrank{استاد}
% نام داوران داخلی و خارجی خود را وارد نمایید.
\internaljudge{دکتر الهام کردی قصردشتی}
\internaljudgerank{استاد}
\externaljudge{دکتر محمدعلی نعمت بخش}
\externaljudgerank{استاد}
\externaljudgeuniversity{دانشگاه داور خارجی}
% نام نماینده کمیته تحصیلات تکمیلی در دانشکده \ گروه
\graduatedeputy{دکتر فریا نصیری مفخم}
\graduatedeputyrank{استاد}
% نام دانشجو را وارد کنید
\name{مهرداد}
% نام خانوادگی دانشجو را وارد کنید
\surname{قصابی}
% شماره دانشجویی دانشجو را وارد کنید
\studentID{4023614029}
% تاریخ پایان‌نامه را وارد کنید
\thesisdate{بهمن ۱۴۰۴}
% به صورت پیش‌فرض برای پایان‌نامه‌های کارشناسی تا دکترا به ترتیب از عبارات «پروژه»، «پایان‌نامه» و «رساله» استفاده می‌شود؛ اگر  نمی‌پسندید هر عنوانی را که مایلید در دستور زیر قرار داده و آنرا از حالت توضیح خارج کنید.
%\projectLabel{پایان‌نامه}

% به صورت پیش‌فرض برای عناوین مقاطع تحصیلی کارشناسی تا دکترا به ترتیب از عبارت «کارشناسی»، «کارشناسی ارشد» و «دکتری» استفاده می‌شود؛ اگر نمی‌پسندید هر عنوانی را که مایلید در دستور زیر قرار داده و آنرا از حالت توضیح خارج کنید.
%\degree{}
%%%%%%%%%%%%%%%%%%%%%%%%%%%%%%%%%%%%%%%%%%%%%%%%%%%%
%% پایان‌نامه خود را تقدیم کنید! %%
\dedication
{
	{\Large تقدیم به:}\\
	\begin{flushleft}{
			\huge
			پدرم که در طول تحصیل پشتیبانم بوده است\\
		}
	\end{flushleft}
}
%% متن قدردانی %%
%% متن قدردانی %%
%% ترجیحا با توجه به ذوق و سلیقه خود متن قدردانی را تغییر دهید.
\acknowledgement{
 دستان پدر و مادر نازنینم را به پاس مهر بیکرانشان به گرمی میفشارم و از استاد راهنما خود جناب آقای دکتر حمیدرضا برادران بابت راهنمایی هایشان در طول انجام این پایان نامه سپاس گزاری میکنم.
}
%%%%%%%%%%%%%%%%%%%%%%%%%%%%%%%%%%%%
%چکیده پایان‌نامه را وارد کنید
\fa-abstract{
استفاده از هوش مصنوعی در پاسخگویی به سوالات پزشکی به عنوان یکی از حوزه‌های نوظهور و مهم در فناوری و بهداشت شناخته می‌شود که در سال‌های اخیر مورد توجه گسترده‌ای قرار گرفته است. این فناوری پیشرفته، با قابلیت‌های ویژه خود، می‌تواند کیفیت خدمات پزشکی ارائه‌شده به بیماران را به شکل چشمگیری ارتقا دهد. همچنین، با سرعت بخشیدن به فرآیند ارائه اطلاعات پزشکی و ارائه پاسخ‌های سریع و دقیق به سوالات پزشکان و بیماران، نقش مهمی در کاهش فشار کاری پزشکان ایفا می‌کند. به این ترتیب، هوش مصنوعی نه تنها موجب افزایش کارایی در سیستم‌های بهداشتی می‌شود، بلکه تجربه کلی بیماران را بهبود می‌بخشد و زمینه ارائه درمان‌های بهتر و مؤثرتر را فراهم می‌کند.

از طرف دیگر از آنجا که پزشکی مبتنی بر استدلال و تحلیل‌های منطقی است، توسعه یک مدل پزشکی که بر پایه زنجیره‌ای از افکار و استدلال‌های منطقی طراحی شده باشد، می‌تواند دقت و کارایی این مدل را به طور قابل توجهی افزایش دهد. چنین رویکردی امکان انجام فرآیندهای پیچیده تشخیصی و درمانی را به صورت ساختاریافته‌تر و هدفمندتر فراهم می‌کند. در این زمینه، هر مرحله از تشخیص و درمان باید مبتنی بر شواهد علمی و داده‌های معتبر باشد. به عنوان مثال، پزشکان در فرآیند تشخیص بیماری‌ها معمولاً از تاریخچه پزشکی، علائم بالینی و نتایج آزمایش‌ها بهره می‌گیرند. با طراحی یک مدل منطقی، این داده‌ها می‌توانند در قالب یک زنجیره منطقی به یکدیگر متصل شوند که به شناسایی الگوها و روابط میان علائم و بیماری‌ها کمک می‌کند.
}
% کلمات کلیدی پایان‌نامه را وارد کنید
\keywords{هوش مصنوعی در پزشکی، مدل های زبانی فارسی، مدل های زبانی پزشکی،  پردازش زبان های طبیعی، توانایی استدلال هوش مصنوعی}
% انتهای وارد کردن فیلد‌ها
%%%%%%%%%%%%%%%%%%%%%%%%%%%%%%%%%%%%%%%%%%%%%%%%%%%%%%
