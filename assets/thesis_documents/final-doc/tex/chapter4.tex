% !TeX root=../main.tex
\chapter{جمع آوری دادگان}
\section{مقدمه}
همان‌طور که پیشتر اشاره شد، در حوزه زبان فارسی نه مدل‌های عمومی موجود هستند و نه مجموعه داده‌های مناسب برای استفاده در پژوهش‌های مرتبط. بنابراین، برای پیشبرد این پایان‌نامه، ناچار به جمع‌آوری دادگان اختصاصی بودیم تا بتوانیم نیازهای تحقیقاتی را برآورده کنیم. فرآیند جمع‌آوری دادگان شامل روش‌هایی مانند ترجمه
\footnote{
	ترجمه می‌تواند به صورت ماشینی یا انسانی انجام شود.
}
داده‌های موجود از زبان‌های دیگر و خزش داده‌ها از منابع مختلف برای ایجاد یک مجموعه داده جامع و کاربردی بوده است.

\section{معرفی پیکره پزشکی فارسی}
عدم وجود یک پیکره پزشکی اختصاصی به زبان فارسی، چالشی قابل توجه برای پژوهشگران و توسعه‌دهندگانی ایجاد می‌کند که هدفشان توسعه مدل های پزشکی در زبان فارسی است. بدون داده‌های متنی باکیفیت و تخصصی که برای آموزش مدل‌های هوش مصنوعی ضروری است، این تلاش‌ها ممکن است با موانع روبه‌رو شوند و در نهایت بر توسعه فناوری‌ها و راه‌حل‌های پیشرفته پزشکی مناسب برای جمعیت فارسی‌زبان تاثیر بگذارند. برای حل این مشکل، ما یک مجموعه داده جامع شامل تقریبا نود میلیون توکن و حدود صد هزار مقاله گردآوری کرده‌ایم.
\footnote{
	برای بازدید از این پیکره میتوانید به آدرس
	\lr{huggingface.co/datasets/gaokerena/medical\_corpus}
	مراجعه کنید
}

گارسیا فررو و همکاران
\cite{b26}
مجموعه‌ای از متون پزشکی را که به چهار زبان (انگلیسی، فرانسوی، اسپانیایی و ایتالیایی) اختصاص داشت، گردآوری کردند که می‌توان آن را همانطور که در جدول 
\ref{corpus_comparison}
نشان داده شده است با مجموعه ما مقایسه کرد.پیکره ای که ما گردآوری کرده ایم از مجله های آنلاین پزشکی خزش شده است که میتوانید سهم هر مجله در این پیکره را در تصویر 
\ref{fig1}
ببینید.
\begin{table}[ht]
	\centering
	\begin{tabular}{|l|c|c|}  % Using vertical lines for a simple table
		\hline
		زبان& تعداد پرسش و پاسخ ها  & گردآورنده \\ \hline
		انگلیسی & \lr{1.1B} & \lr{I. Garcia Ferrero et al.} \\ \hline
		اسپانیایی & \lr{950M} & \lr{I. Garcia Ferrero et al.}  \\ \hline
		فرانسوی & \lr{675M} &  \lr{I. Garcia Ferrero et al.}  \\ \hline
		ایتالیایی& \lr{143M} &  \lr{I. Garcia Ferrero et al.}  \\ \hline
		فارسی& \lr{90M} & ما	\\ \hline
	\end{tabular}
	\caption{مقایسه پیکره گردآوری شده با پیکره های گردآوری شده توسط
		\lr{I. Garcia Ferrero et al.}
	}
	\label{corpus_comparison}
\end{table}

\section{معرفی مجموعه داده 
	\lr{MF3QA}
}
گردآوری یک مجموعه داده واقعی از پرسش‌ وپاسخ‌های پزشک و بیمار 
اهمیت بسیاری در ارتقا توانایی‌های مدل‌های زبانی در حوزه بهداشت و درمان دارد. چنین مجموعه داده‌ای به مدل‌ها امکان می‌دهد تا اطلاعات ارزشمندی را که از تعاملات واقعی میان ارائه‌دهندگان خدمات بهداشتی و بیماران به دست می‌آید، بیاموزند. با تحلیل این تعاملات واقعی، مدل‌های زبانی می‌توانند به درک جزئیات اصطلاحات پزشکی، نگرانی‌های بیماران، و زمینه پیرامون سؤالات بهداشتی دست یابند. علاوه بر این، این مجموعه داده مدل‌ها را قادر می‌سازد نه تنها محتوای دقیق پاسخ‌ها، بلکه ساختار و لحن مناسب برای پاسخ‌دهی به سوالات را نیز یاد بگیرند. این فرآیند دوگانه یادگیری از اهمیت بالایی برخوردار است، زیرا به مدل امکان می‌دهد پاسخ‌هایی دقیق، همدلانه و متناسب با زمینه ارائه دهد و در نهایت ارتباط و پشتیبانی از بیماران در محیط‌های پزشکی را بهبود بخشد.

در این زمینه، یانگ لیو در مقاله مروری
\footnote{\lr{survey}}
خود 
\cite{b27}
به چندین مجموعه داده واقعی پرسش‌ وپاسخ پزشک وبیمار اشاره کرده است، مقایسه‌ای میان این مجموعه‌ دادگان و مجموعه داده ما در جدول 
\ref{tab:mf3qa_comparison}
ارائه شده است.


\begin{table}[ht]
	\centering
	\begin{tabular}{|l|c|c|c|}  % Using vertical lines for a simple table
		\hline
		نام مجموعه داده            &زبان & تعداد پرسش و پاسخ ها & گردآورنده \\ \hline
		\lr{ChatDoctor} 	   &انگلیسی	 & \lr{100K}       & \lr{Yunxiang Li et al.}
		\cite{b15}
		\\ \hline
		\lr{CMtMedQA}  	   &چینی	 & \lr{68K}        &  \lr{Songhua Yang et al.}
		\cite{b28}
		\\ \hline
		\lr{DISC-Med-SFT}	   &چینی	 & \lr{465K}       & \lr{Zhijie Bao et al.}   
		\cite{b29}
		\\ \hline
		\lr{HuatuoGPT-}	   &چینی	 & \lr{226K}        & \lr{Hongbo Zhang et al.}  
		\cite{b30}
		\\ 
		\lr{sft-data-v1}	   &	 &         & 
		\\ \hline
		\lr{Huatuo-26M}	   &چینی	 & \lr{26M}         & \lr{Jianquan Li et al.}
		\cite{b31}
		\\ \hline
		\lr{MedDialog}       &چینی و انگلیسی &\lr{3.66M}      & \lr{Guangtao Zeng et al.}\cite{b32}  \\ \hline
		\lr{Medical-Meadow}  &انگلیسی  & \lr{160k}        & \lr{Tianyu Han et al.}
		\cite{b33}
		\\ \hline
		\lr{MF3QA}           &فارسی  & \lr{20k}           & ما \\ \hline
	\end{tabular}
	\caption{مقایسه مجموعه داده های پرسش و پاسخ آزاد پزشکی با مجموعه داده گردآوری شده
	}
	\label{tab:mf3qa_comparison}
\end{table}


\begin{figure}[ht]
	\centerline{\includegraphics[width=0.7\textwidth]{fig1}}
	\caption{
		سهم هر مجله در پیکره پزشکی فارسی گردآوری شده
	}
	\label{fig1}
\end{figure}

\subsection{
	منابع مجموعه داده
	\lr{MF3QA}
}
همان‌طور که در شکل
\ref{fig2}
نشان داده شده است، برای گردآوری مجموعه داده
\lr{MF3QA}
مراحل مختلفی طی شده است. در بخش آموزش، پرسش‌وپاسخ‌های بیمار و پزشک موجود در تالارهای گفت‌وگوی پزشکی فارسی
\footnote{\lr{Persian medical forums}}
"دکترهست" و "نی‌نی‌بان" را خزش کرده‌ایم. برای بخش اعتبارسنجی، تنها از داده‌های موجود در سایت "نی‌نی‌بان" استفاده کرده‌ایم تا انسجام بیشتری در این بخش حاصل شود. در بخش آزمایش نیز، از سایت‌های "دکتریاب" و "ایزوویزیت" بهره برده‌ایم و به‌منظور اطمینان از تنوع داده‌ها، مجموعه داده پرسش‌وپاسخ 
\lr{K-QA}
\cite{b12}
را ترجمه کرده و به این بخش اضافه کرده‌ایم.
\begin{figure}[ht]
	\centerline{\includegraphics[width=0.9\textwidth]{fig2}}
	\caption{
		سهم هر تالار گفتگو در مجموعه داده
		\lr{MF3QA}
	}
	\label{fig2}
\end{figure}

\subsection{
	فیلتر کردن رکورد های مجموعه داده
	\lr{MF3QA}
}
در پایان‌نامه حاضر، بیش از صد و هشتاد هزار جفت پرسش‌وپاسخ از تالارهای گفت‌وگوی پزشکی فارسی گردآوری شده است. این جفت‌های پرسش‌وپاسخ، چه به صورت دستی
\footnote{
	فرآیند فیلتر کردن دستی توسط خانم زهرا کاظمی و آقای میلاد توکلی، از دانشجویان کارشناسی مهندسی کامپیوتر، انجام شده است.
}
و چه به صورت خودکار، مورد بررسی قرار گرفته و جفت‌هایی که حاوی اطلاعات مفید نبودند، حذف شده‌اند.
\footnote{
	برای بازدید از مجموعه داده 
	\lr{MF3QA}
	به آدرس
	\lr{huggingface.co/datasets/gaokerena/MF3QA}
	و برای بازدید از صد و هشتاد هزار جفت پرسش‌وپاسخ خزش شده به آدرس
	\lr{huggingface.co/datasets/gaokerena/MF3QA\_uncleaned}
	مراجعه کنید.
}

این رویکرد مشابه کاری است که یونشیانگ لی و همکارانش برای توسعه مدل زبانی
\lr{Chat Doctor}
انجام داده‌اند.
\cite{b16}
آنها نیز داده‌ها را از تالارهای گفت‌وگوی پزشکی انگلیسی استخراج کرده و نیمی از جفت‌های پرسش‌وپاسخ را بر اساس طول پاسخ‌ها کنار گذاشته‌اند
\footnote{
	فیلتر کردن آنها صرفا بر اساس طول پاسخ بوده ولی همانطور که پیشتر اشاره شد ما برای فیلتر کردن از روش های دستی نیز استفاده کرده ایم.
}
، 
چراکه پاسخ‌های کوتاه‌تر معمولا حاوی اطلاعات مفیدی نیستند. با این حال، ما با چالش بزرگ‌تری مواجه بودیم؛ پزشکان فارسی‌زبان معمولا پاسخ‌های بسیار کوتاه‌تری نسبت به همتایان انگلیسی خود ارائه می‌دهند. این امر ما را مجبور کرد تا بیش از هشتاد درصد از رکوردهای پرسش‌وپاسخ خود را برای تضمین کیفیت کنار بگذاریم.

\subsubsection{
	خزش از تالار گفتگو دکترهست
}
خزش از تالار گفتگوی “دکترهست”، که اصلی‌ترین منبع مجموعه داده
\lr{MF3QA}
است، با چالش خاصی همراه بود. این تالار گفتگو تمام رکوردهای تعامل پزشک و بیمار خود را به صورت مستقیم در سایت ارائه نمی‌دهد و فقط به دو هزار رکورد آخر دسترسی می‌دهد. علاوه بر این، هر رکورد به صد رکورد مرتبط دیگر پیوند داده شده است.

برای حل این چالش، از الگوریتم
\ref{alg:BFS}
استفاده شد. در این روش، داده‌های تالار گفتگو به صورت یک گراف در نظر گرفته شده و با استفاده از جستجوی عرض-اول 
\footnote{\lr{breadth first search}}
توانستیم حدود صد و بیست هزار رکورد از مجموع دویست هزار رکورد موجود در این تالار گفتگو را استخراج کنیم. این فرایند حدود دو هفته طول کشید.

\begin{algorithm}[ht]
	\onehalfspacing
	\caption{
		الگوریتم جستجو اول عرض برای استخراج رکورد های پرسش و پاسخ پزشکی
	} \label{alg:BFS}
	\begin{algorithmic}[1]
		\REQUIRE گره های دارای دسترسی در تالار گفتگو (برگ ها)		
		\\
		\ENSURE مجموعه‌ای از گره‌های بازدید شده
		\STATE یک پشته خالی $S$ ایجاد کن
		\STATE یک مجموعه خالی $Visited$ ایجاد کن
		\STATE گره مبدأ $v$ را به پشته $S$ اضافه کن
		\WHILE{پشته $S$ خالی نیست}
		\STATE یک گره $u$ را از پشته $S$ بردار
		\IF{گره $u$ بازدید نشده است}
		\STATE گره $u$ را به مجموعه $Visited$ اضافه کن
		\FOR{هر همسایه $n$ از گره $u$}
		\IF{گره $n$ بازدید نشده است}
		\STATE گره $n$ را به پشته $S$ اضافه کن
		\ENDIF
		\ENDFOR
		\ENDIF
		\ENDWHILE
		\RETURN نود های بازدید شده
		
	\end{algorithmic}
\end{algorithm}

\section{ترجمه قسمت پزشکی مجموعه داده
	\lr{MMLU}
}
مجموعه داده 
\lr{MMLU}
\cite{b10}
\footnote{\lr{Massive Multitask Language Understanding}}
یکی از معتبرترین مجموعه داده‌ها برای ارزیابی توانایی مدل‌های زبانی در درک و پاسخ‌دهی به سوالات چندوظیفه‌ای است. این مجموعه شامل سوالاتی در موضوعات مختلف از جمله علوم پزشکی، مهندسی، علوم انسانی و دیگر حوزه‌ها است که به‌صورت چندگزینه‌ای طراحی شده‌اند.

در پروژه ما، برای ارزیابی مدل زبانی پزشکی توسعه‌یافته، بخش پزشکی این مجموعه داده را به زبان فارسی ترجمه کردیم.
\footnote{
ترجمه توسط آقای امیرحسین پورسینا دانشجوی پزشکی انجام شده است.
}
 هدف از این کار، تطبیق داده‌های ارزیابی با زبان مورد استفاده در مدل و بررسی توانایی مدل در پاسخ‌دهی دقیق به سوالات تخصصی پزشکی در زبان فارسی بود.
\footnote{
برای بازدید از ترجمه این مجموعه داده به آدرس
\lr{huggingface.co/datasets/gaokerena/FA\_MED\_MMLU}
مراجعه کنید.
 }
\section{گردآوری سوالات کنکور علوم پایه پزشکی ایران
}
آزمون علوم پایه پزشکی یک آزمون سراسری در ایران است که دانشجویان پزشکی موظف هستند پس از گذراندن دروس علوم پایه معمولا در پنج ترم در آن شرکت کنند. این آزمون به منظور سنجش میزان آموخته‌های دانشجویان از دروس علوم پایه و آمادگی آنها برای ورود به مراحل بالینی برگزار می‌شود. در صورتی که دانشجو پس از سه مرتبه در این آزمون قبول نشود
\footnote{نمره قبولی در این آزمون در سالهای مختلف متفاوت است و معمولا حدود سی و شش درصد میباشد}
از ادامه تحصیل در رشته پزشکی محروم میشود.

برای سنجش دانش پزشکی مدل زبانی خود ما سوالات این آزمون را از 
\lr{pdf}
سوالاتی که سازمان سنجش برای آن منتشر میکند استخراج کرده ایم.
\footnote{برای این کار از کتابخانه
\lr{fitz}
پایتون استفاده شده است.
}
\footnote{
	برای بازدید از این مجموعه داده به آدرس
	\lr{https://huggingface.co/datasets/gaokerena/KOPP}
	مراجعه کنید.
}
\section{ترجمه ماشینی مجموعه داده
	\lr{MedMCQA}
}