% !TeX root=../main.tex

\chapter{دیباچه}
\section{هدف پژوهش}
هدف از این پژوهش، توسعه یک مدل زبانی پزشکی فارسی بر پایه استدلال 
\footnote{\lr{reasoning}}
است که قابلیت اجرا روی دستگاه‌های محلی را داشته باشد. اجرا روی دستگاه‌های محلی از آن جهت حائز اهمیت است که داده‌های پزشکی اغلب حساس و خصوصی هستند و ارسال آنها به سرورهای خارجی ممکن است خطرات جدی برای حریم خصوصی بیماران ایجاد کند.
\section{کاربرد پژوهش}
\subsection{کاربرد مدل های زبانی پزشکی}
مدل‌های زبانی در سال‌های اخیر با استفاده از داده‌های بسیار گسترده‌تر و معماری‌های پیشرفته‌تر به پیشرفت‌های چشمگیری دست یافته‌اند. این مدل‌ها توانایی درک بهتر مفاهیم، تولید متن‌های طبیعی‌تر و پاسخ‌دهی دقیق‌تر به سؤالات را پیدا کرده‌اند. 

 این پیشرفت‌ها منجر به افزایش چشمگیر کاربرد هوش مصنوعی
\footnote{\lr{artificial intelligence}}
در حوزه‌های مختلف، به‌ویژه در زمینه پزشکی، شده است.امروزه در حوزه پزشکی، مدل‌های زبانی مبتنی بر یادگیری ژرف
\footnote{\lr{deep learning}}
نقش مهمی در تحلیل داده‌های پزشکی، بهبود دقت تشخیص بیماری‌ها، ارائه پیشنهادهای درمانی دقیق‌تر و افزایش کیفیت مراقبت از بیماران ایفا می‌کنند. علاوه بر این، این فناوری، به بهینه‌سازی سیستم‌های اداری و کاهش بار کاری کادر درمانی کمک شایانی کرده است. به عنوان مثال، مدل‌های هوش مصنوعی قادرند با تحلیل داده‌های حاصل از پرونده‌های پزشکی، الگوهای مرتبط با بیماری‌ها را شناسایی کنند و اطلاعات ارزشمندی را برای تصمیم‌گیری سریع‌تر و دقیق‌تر در اختیار پزشکان قرار دهند. 
این مدل‌ها همچنین می‌توانند نقش مهمی در تکمیل مشاوره‌های پزشکی ایفا کرده و به پزشکان در ارائه اطلاعات دقیق‌تر و سریع‌تر کمک کنند.و حتی شاید در آینده ای نه چندان دور بتوانند جای پزشکان را در مشاوره های پزشکی بگیرند.

این تحول نه تنها به افزایش کارایی و بهره‌وری در سیستم‌های درمانی منجر شده است، بلکه تجربه کلی بیماران را نیز بهبود بخشیده و امکان ارائه خدمات درمانی بهتر و مؤثرتر را فراهم کرده است. به همین دلیل، توسعه و استفاده از مدل های زبانی پزشکی
\footnote{\lr{medical language models}}
، همچنان مورد توجه پژوهشگران و متخصصان قرار دارد.
\subsection{کاربرد مدل های زبانی پزشکی فارسی}
علیرغم پیشرفت‌های چشمگیر در توسعه مدل‌های زبانی پزشکی به زبان انگلیسی، در حوزه زبان فارسی هنوز کار چندانی صورت نگرفته است. این در حالی است که در سرتاسر جهان میلیون ها نفر تنها قادر به استفاده از این زبان هستند؛
بنابراین تلاش برای توسعه یک مدل زبانی پزشکی در زبان فارسی میتواند گامی رو به جلو در ارتباطات و خدمات درمانی کشور های فارسی زبان باشد. 
\section{مراحل انجام پایان نامه}
همانطور که در جدول 
\ref{table:phases_information}
این پایان‌نامه در دو فاز اصلی طراحی و اجرا شده است. فاز نخست به جمع‌آوری دادگان پزشکی فارسی و توسعه مدلی با نام 
گائوکرنا-\lr{V} 
اختصاص دارد که فاقد توانایی استدلال بوده و بیشتر بر درک سیستم یک 
\footnote{
در علم رفتارشناسی به درک سریع، شهودی و بدون نیاز به تفکر ژرف درک سیستم یک و به درک آهسته، غیر شهودی و نیازمند استدلال درک سیستم دو میگویند.
}
زبان تمرکز دارد. از این فاز، مقاله‌ای با عنوان "اهرم قرار دادن داده‌های آنلاین برای بهبود دانش پزشکی یک مدل زبانی کوچک پزشکی فارسی" 
استخراج شده است که به تشریح فرآیند جمع‌آوری داده‌ها و نحوه بهینه‌سازی دانش پزشکی مدل می‌پردازد. در فاز دوم این پژوهش ابتدا تکنیک های جدیدی برای ارتقای توانایی استدلال و درک سیستم دو مدل معرفی شده و سپس مدل 
گائوکرنا-\lr{R} 
در این فاز توسعه داده شده است.
از این فاز نیز، مقاله‌ای با عنوان "؟"
 استخراج شده است.
 
\begin{table}[h!]
	\centering
	\begin{tabular}{|c|c|c|}
		\hline
		\textbf{\lr{gaokerena-R}}& \textbf{\lr{gaokerena-V}} &   \\ \hline
		\lr{mehrdadghassabi/gaokerena-R}   &     \lr{mehrdadghassabi/gaokerena-V}  & مخزن گیت هاب      \\ \hline
		\lr{gaokerena/gaokerena-r1.0}    &     \lr{gaokerena/gaokerena-v1.0}  & مخزن پارامتر ها      \\ \hline
		\lr{https://arxiv.org/pdf/0000.00000}  &    \lr{https://arxiv.org/pdf/2505.16000}   & پیوند مقاله      \\ \hline
		300 دلار  &      300 دلار   & هزینه     \\ \hline
		بله  &     خیر   & توانایی استدلال     \\ \hline
		دکتر حمیدرضا برادران،پدرام رستمی &   دکتر حمیدرضا برادران،پدرام رستمی،   & همکاران    \\
		و صدرا حکیم &   میلاد توکلی، امیرحسین پورسینا   &    \\ 
		&   و زهرا کاظمی   &    \\ \hline
	\end{tabular}
	\caption{اطلاعات دو فاز پایان نامه}
	\label{table:phases_information}
\end{table}

\section{ساختار پایان نامه}
در این پایان‌نامه، ساختار فصل‌ها به گونه‌ای طراحی شده است که مراحل مختلف پژوهش به صورت منظم و هدفمند ارائه شوند. فصل دوم به بررسی کارهای پیشین اختصاص دارد که در آن مطالعات انجام‌شده در زمینه‌های مرتبط مرور خواهند شد. در فصل سوم، به دلیل عدم وجود دادگان پزشکی در حوزه زبان فارسی، فرآیند گرد‌آوری و آماده‌سازی این دادگان به‌طور دقیق تشریح خواهد شد. سپس در فصل چهارم، با استفاده از دادگان معرفی‌شده در فصل سوم، مدل اولیه با نام گائوکرنا-\lr{V}
\footnote{
نام گائوکرنا از درختی افسانه‌ای الهام گرفته شده است که در روایات اساطیری زرتشتی به ‌عنوان نماد شفادهی و جاودانگی شناخته می‌شود.
}
 معرفی و تحلیل می‌شود. در فصل پنجم، توانایی‌های استدلال در مدل‌های هوش مصنوعی مورد بررسی قرار گرفته و چالش‌ها و راهکارهای مرتبط با استدلال ارائه خواهند شد. در ادامه، در فصل ششم، با معرفی تکنیک هایی برای بهبود توانایی استدلال یک مدل زبانی مدل پیشرفته‌تری به نام گائوکرنا-\lr{R} معرفی و ویژگی‌های آن به‌تفصیل شرح داده می‌شود. در نهایت، فصل پایانی به جمع‌بندی نتایج پژوهش و پیشنهاداتی برای تحقیقات آینده اختصاص دارد.
